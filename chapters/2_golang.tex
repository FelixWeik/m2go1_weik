\chapter{M2Go Part 2: Golang}
\label{cha:golang}
%//TODO am Schluss nochmal drübergehen und checken, dass kein "we" oder "us" benutzt wird
This section introduces the Golang programming language.
It is a compiled, statically typed language developed by Google, commonly used for network and web applications.
Its performance and simplicity make it a good choice for microservices, APIs and other network applications.

The first section is about setting up the Golang environment and cloning the gitlab repository.
\autoref{sec:basic_go_program} will introduce the basic syntax of Golang by implementing a simple hello world program.
\autoref{sec:go_program_car_rental} will manifest the basic Golang knowledge by analysing the structure and requirements of the CarRental program.
Finally, \autoref{sec:advanced_go_program_car_rental_cli} will introduce advanced Golang features by implementing a command line interface for the CarRental program.

%==============================================================================

\section{M2Go Environment Installation}
\label{sec:m2go_env_installation}
This section will provide guidance on how to install the Golang environment and how to clone the gitlab repository.

\label{sec:exercise_install_go_env}

\subsection{Exercise M2GoEnvironmentInstallation}
\label{sec:exercise_m2go_environment_installation}

\subsubsection*{Environment Description}
The hardware used is a Windows Surface Studio with an 11th Gen Intel Core i5-11300H, 16 GB of RAM, and 256 GB of SSD storage.
The operating system is Windows 11 64-bit. However, the installation is carried out on WSL 2 running Ubuntu 20.04.6 LTS installed on the machine.

\subsubsection*{Installation Steps}
WSL2 (Ubuntu 20.04.6 LTS), VSCode (1.84.2) and Git (2.25.1) are already installed on the machine. 
Therefore no further installation steps are required for these tools.
The correct environment-installation on the author's machine can be seen in \autoref{fig:screendump_installation}.

Carrying out the installation of Go, version go1.21.4 is installed.
To install Go on the machine, the following steps are required:
\begin{enumerate}
    \item Download the latest version of Go from \url{https://go.dev/dl/go1.21.4.linux-386.tar.gz} by using  \texttt{wget} in the terminal.
    \item Extract the downloaded archive to \texttt{/usr/local} via \texttt{sudo tar -C /usr/local -xzf go1.21.3.linux-386.tar.gz}
    \item Under \texttt{/etc/profile} add the following lines to the bottom of the file: 
    \begin{itemize}
        \item \texttt{export PATH=\$PATH:/usr/local/go/bin}
        \item export \texttt{GOPATH=\$HOME/go}
    \end{itemize}
    \item Now reload the terminal or force the update by running \texttt{source \$HOME/.profile}
    \item Verify the installation by running \texttt{go version}
\end{enumerate}

\begin{figure}[H]
	\centering
	\includegraphics[width=0.5\textwidth]{figures/goLang/installation_screendump.png}
	\caption{Screendump showing the correct Installation of the Environment}
	\label{fig:screendump_installation}
\end{figure}

Further details on the installation and especially the path variables can be found in \autoref{sec:go_environment_variables} on path variables in Ubuntu.
\label{sec:exercise_cm_gitlab_usage}

\subsection{Exercise CMGitLabUsage}

\subsubsection*{Description of the M2Go Subgroup Structure}
The subgroup structure in \autoref{fig:screendump_subgroupStructure} only contains the Golang repository.
As we advance in the course, more repositories will be added to this subgroup.
This repository consists out of the following three folders:
\begin{enumerate}
    \item BasicGoProgram/HelloWorld: This folder contains the hello world program introducing the basic Golang syntax
    \item CarRental: This folder contains the CarRental program introducing general Golang features
    \item CarRentalCLI: This folder contains the CarRentalCLI program implementing a more complex version of the CarRental program
\end{enumerate}
Further more there is the .gitignore file and the README.md file.
This structure can be further examined in \autoref{fig:screendump_subsubgroupStructure}.

\begin{figure}[h]
    \centering
    \includegraphics[width=0.8\textwidth]{figures/goLang/golang_personalSubgroupStructure.png}
    \caption{Screendump showing the subgroup structure}
    \label{fig:screendump_subgroupStructure}
\end{figure}

\begin{figure}
    \centering
    \includegraphics[width=0.8\textwidth]{figures/goLang/golang_personalSubsubgroupStructure.png}
    \caption{Screendump showing the subgroup structure of the Golang folder}
    \label{fig:screendump_subsubgroupStructure}
\end{figure}

\subsubsection*{Repository Cloning Steps}
To clone the repository, the following steps are required:
% //TODO über ssh
\begin{enumerate}
    \item Create a Personal Access Token (PAT) on GitLab by going to your Profile Settings and then to Access Tokens
    \item Save the PAT in a safe place
    \item Go to the repository and copy the HTTPS clone URL
    \item Open the WSL2 terminal and navigate to the folder where you want to clone the repository, in my case \texttt{/home/felix/WASA\_M2Go}
    \item Download the repository by running the following command in the terminal: \texttt{git clone <HTTPS clone URL>}
\end{enumerate}

\subsubsection*{First Commit}
After changing the placeholder text in the README.md file, the first commit is done by using the graphical features Visual Studio Code offers.
The commit message shown in \autoref{fig:screendump_readmeCommitMessage} is the following: \texttt{Update author and supervisor in README.md}.

After committing the changes, the changes are pushed to the remote repository, which then become visible as shown in \autoref{fig:screendump_readme}.

\begin{figure}[h]
    \centering
    \includegraphics[width=0.8\textwidth]{figures/goLang/golang_screendumpReadmeCommit.png}
    \caption{Screendump showing commit message and the changed files}
    \label{fig:screendump_readmeCommitMessage}
\end{figure}

\begin{figure}[h]
    \centering
    \includegraphics[width=0.5\textwidth]{figures/goLang/golang_screendumpReadme.png}
    \caption{Screendump showing the updated README.md file}
    \label{fig:screendump_readme}
\end{figure}


%==============================================================================

\section{Basic Go Program HelloWorld}
\label{sec:basic_go_program}
This section implements a simple hello world program, introdcing the basic syntax of Golang.


\section{Basic Hello World Program}
\label{sec:basic_hello_world_program}

\subsection*{Run Hello World}
After following the instruction in Mic-Con, the code is executed by running \texttt{go run main.go} in the terminal.
The code was executed twice: The first time the given code was executed leading to the output shown in figure \ref{fig:screendump_helloWorld_basicExecution}.
The second time the code was executed with different text, which can be seen in figure \ref{fig:screendump_helloWorld_differentText}.

\begin{figure} [h]
    \centering
    \includegraphics[width=0.8\textwidth]{figures/goLang/helloWorld/golang_helloWorld_basicExecution.png}
    \caption{Screendump showing the basic execution of the Hello World program}
    \label{fig:screendump_helloWorld_basicExecution}
\end{figure}

\begin{figure}[h]
    \centering
    \includegraphics[width=0.8\textwidth]{figures/goLang/helloWorld/golang_helloWorld_ExecutionDifferentText.png}
    \caption{Screendump showing the execution of the Hello World program with different text}
    \label{fig:screendump_helloWorld_differentText}
\end{figure}

\subsection*{Code Explanation}
\begin{lstlisting}[
    language=Golang,
    caption={Hello World Program in Golang with explanations},
    label={lst:helloWorld},
    numbers=left,
    numberstyle=\tiny
    ]
    // main.go 
    // Author: Felix Weik

    package main    // package declaration: every executable belongs to the main package
                    // by declaring this package, a executable file is produced after compliation
    
    import "fmt"    // import the fmt package, a standard library package 
                    // implementing formatted I/O functions
                    // after import, one can use the functions of the imported package

    func main() {   // declares the function main, which is the 
                    // entry point of the program
        fmt.Println("Hello, World!")    // call the Println function 
                                        // of the fmt package; Println prints 
                                        // the given text to the standard output
    }               // end of the main function
    
\end{lstlisting}

\subsection*{HelloName}
The goal of this task is to extend the already given HelloWorld program to the HelloName program, prompting the user for a name then showing the given prompt in the output.
The code is shown in listing \ref{lst:helloName}:
\begin{lstlisting}[
    language=Golang,
    caption={Extension of helloWorld to helloName},
    label={lst:helloName},
    numbers=left,
    numberstyle=\tiny
    ]
    // main.go
    // Author: Felix Weik

    package main

    import (
        "fmt"
    )

    func main() {
        greet()
    }

    func greet() {
        var name string
        fmt.Print("What is your name? ")
        fmt.Scanln(&name)
        fmt.Println("Hello, " + name + "!")
    } 
\end{lstlisting}

The result of the executed code is shown in figure \ref{fig:screendump_helloName}

\begin{figure}
    \centering
    \includegraphics[width=0.8\textwidth]{figures/goLang/helloWorld/golang_helloWorld_helloName.png}
    \caption{Screendump showing the execution of the HelloName program}
    \label{fig:screendump_helloName}
\end{figure}

%==============================================================================

\section{Go Program CarRental}
\label{sec:go_program_car_rental}
This section introduces fundamental programming concepts in Golang.
First, the requirements analysis of the CarRental program is conducted, by reviewing an informal story in section \ref{sec:exercise_use_case_diagram_car_rental}.
The following design-phase in section \ref{cha:design_of_data_and_functionality} spepcifies data structures and the functionality of the use case.
The last section introduces concepts of implementing and testing the design artifacts.

\subsection{Excercise UseCaseDiagram CarRental}
\label{sec:exercise_use_case_diagram_car_rental}
The general functionality of the application CarRental is outlined by Alice's Car Rental story as described in \cite{CM-T-GO} listing 3.1.
It provides a general overview of the application and the user flow of registering and renting a car.

By using the story, the use cases and actors can be derived.
The following tasks are based on the story and help derive the use cases, actors and services modeled in the use case diagram.
The story is shown in \autoref{fig:car_rental_use_case_diagram}.

\subsubsection*{Derive Use Cases and Actors}
The goal of this task is to find four use cases and one actor according to the given story "Alice's Car Rental" from listing 3.1 in \cite{CM-T-GO}.

\autoref{lst:car_rental_use_case_1} was derived from line 4 of the given story.

In the story Alice rents a car, a VW ID.2, for a certain period.
This is done by entering the rental period and choosing the car from the list of available cars.

After successfully renting the car, the customer is prompted with a rental confirmation and the rental is stored.

\begin{lstlisting}[
    caption={Use Case 1: "Rent a Car"},
    label={lst:car_rental_use_case_1},
    style=kit-cm,
]
Title: Rent a Car
Primary Actor: Customer
Secondary Actor: None

Preconditions:
    - Customer is registered and logged in
    - Customer has a valid driver's license and credit card

Postconditions:
    - The customer has rented a car for the chosen time interval

Flow:
1. The customer calls the application CarRental.
2. The customer enters a start date and end date as the rental period
3. The system shows a list of the cars that are available at the selected rental period
4. The customer selects one of the listed cars
5. The system shows a rental confirmation and stores the rental

Alternative flows:
3a. no cars are available at the selected time interval
    3a1. The system shows a message that no cars are available at the selected time interval and the flow continues from 2 or terminates
4a. The customer does not want to rent any of the offered cars
    4a1. The flow terminates    
\end{lstlisting}

\autoref{lst:car_rental_use_case_2} describes the registration process itself.

It was derived from line 3 of the given story.
In order to use the application, Alice needs to register first.

After successfully registering, Alice can log in to the application and rent a car.

\begin{lstlisting}[
    caption={Use Case 2: "Registering Process"},
    label={lst:car_rental_use_case_2},
    style=kit-cm,
]
Title: Registering Process
Primary Actor: Customer
Secondary Actor: None

Preconditions:
    - The customer is not registered yet
    - Customer has a valid email address
    - Customer has a valid credit card and driver's license
    - Customer is at least 18 years old
    - Customer is not blacklisted by the car rental company or an insurance company

Postconditions:
    - Customer is registered
    - Customer can rent a car
    - Customer can log in to the car rental company's website

Flow:
1. Customer visits the car rental company's website or opens the app 
2. Customer is asked to register or login
3. Customer clicks on "Register"
4. Customer is prompted for email, name, driver's license, credit card number, date of birth
5. A data validation is performed
6. Customer needs to authenticate his email address
7. Customer chooses a password and a username
8. Customer shows a registration confirmation and the customer is registered successfully 

Alternative flows:
3a. Customer is already registered and clicks on "Login" instead of "Register"
5a. Customer prompts false information or leaves nonoptional fields empty, so the data validation fails
    5a1. The customer is asked to fill out the form again
6a. Customer does not authenticate his email address
    6a1. The customer will not be registered and the registration process will be aborted
7a. Customer chooses a username that is already taken
    7a1. The customer is asked to choose another username
7b. The provided password does not fulfill the given criteria
    7b1. The customer is asked to choose a stronger password
\end{lstlisting}

There are two types of cancellation: Cancellation of a rental and cancellation of the registration.
\autoref{lst:car_rental_use_case_3} describes the cancellation of a rental as described in lines 5 and 6 of the given story.

Like Alice in the story, the customer can cancel a rental if he successfully rented a car but does not want to use it anymore.
The customer can cancel the rental as long as the time interval of the rental has not started yet.

After successfully canceling the rental, the customer is prompted with a cancellation confirmation.

\begin{lstlisting}[
    style=kit-cm,
    caption={Use Case 3: "Cancellation of a Rental"},
    label={lst:car_rental_use_case_3},
]
Title: Cancellation of a Rental
Primary Actor: Customer
Secondary Actor: None

Preconditions:
    - Customer has rented a car
    - Customer is logged in
    - The time interval of the rental has not started yet

Postconditions:
    - Customer has cancelled the rental
    - Customer is prompted with a cancellation fee if necessary

Flow:
1. Customer calls the application CarRental
2. Customer clicks on "My Rentals"
3. Customer selects the rental he wants to cancel
4. Customer clicks on "Cancel Rental"
5. Customer is prompted with a cancellation fee if necessary
6. Customer confirms the cancellation
7. Customer is asked to confirm the cancellation again via email
8. Customer confirms the cancellation via email
9. Customer is prompted with a cancellation confirmation

Alternative flows:
3a. Customer has no rentals
    3a1. The system shows a message that the customer has no rentals and the flow terminates
4a. Customer does not want to cancel the rental and the flow terminates
5a. Customer does not want to pay the cancellation fee
    5a1. The flow terminates and the rental is not canceled
8a. Customer does not confirm the cancellation via email
    8a1. The flow terminates and the rental is not canceled
\end{lstlisting}

\autoref{lst:car_rental_use_case_4} describes the cancellation of the registration as described in line 6 of the given story.

In the story, Alice cancels her registration due to personal reasons.
The customer can cancel the registration if he does not want to use the application anymore.

After successfully canceling the registration, the customer is prompted with a cancellation confirmation and can neither log in nor use the application without registering again.

\begin{lstlisting}[
    float=h,
    caption={Use Case 4: "Cancellation of the Registration"},
    label={lst:car_rental_use_case_4},
    style=kit-cm,
]
Title: Cancellation of the Registration
Primary Actor: Customer
Secondary Actor: None

Preconditions:
    - Customer is registered
    - Customer is logged in
    - Customer has no rentals or outstanding payments

Postconditions:
    - Customer is not registered anymore

Flow:
1. Customer calls the application CarRental
2. Customer clicks on "My Account"
3. Customer clicks on "Cancel Registration"
4. Customer confirmes the cancellation
5. Customer is asked to confirm the cancellation again via email
6. Customer confirms the cancellation via email
7. Customer is prompted with a cancellation confirmation

Alternative flows:
3a. Customer has outstanding payments or rentals
    3a1. The system shows a message that the customer has outstanding payments or rentals and the flow terminates
4a. Customer does not want to cancel the registration and the flow terminates
6a. Customer does not confirm the cancellation via email
    6a1. The flow terminates and the registration is not canceled
\end{lstlisting}

\subsection*{Modelling the Use Case Diagram with UMLet}
The installation of the standalone version of UMLet is carried out as follows:
\begin{enumerate}
    \item Download the latest version of UMLet from \url{https://www.umlet.com/changes.htm}
    \item Extract the downloaded archive
    \item Run the executable file \texttt{umlet.sh} in the extracted folder
\end{enumerate}

A screen dump of the folder structure of the extracted archive is shown in \autoref{fig:umlet_folder_structure}.
The use case diagram is shown in \autoref{fig:car_rental_use_case_diagram}.

\begin{figure}
    \centering
    \includegraphics[width=0.8\textwidth]{figures/goLang/carRental/carRental_umletInstallation.png}
    \caption{Folder Structure of the Extracted UMLet Archive}
    \label{fig:umlet_folder_structure}
\end{figure}

\autoref{fig:car_rental_use_case_diagram} illustrates the use cases and their relationships.
If an user wants to rent a car, the user must be a registered customer.
Therefore, \texttt{Customer Registration} is included in the \texttt{Rent A Car} use case.
Following the same logic, one cannot cancel a rental without renting it first.
This explains the extension-relation between \texttt{Rental Cancellation} and \texttt{Rent A Car}.
The same applies for \texttt{Customer Registration} and \texttt{Registration Cancellation}.

\begin{figure}
    \centering
    \includegraphics[width=0.7\textwidth]{figures/goLang/carRental/carRental_umlDiagram.png}
    \caption{Use Case Diagram of the \textbf{Best Rental} Application}
    \label{fig:car_rental_use_case_diagram}
\end{figure}

\subsection{Excercise UseCase RentACar}
\label{sec:exercise_use_case_rent_a_car}
\subsubsection*{Alice's Car Rental}
This task aims to complete the use-case story "Rent a Car" for Alice.
This is done by adding the given data from the story to the use-case story.

The completed use-case story is shown in \autoref{lst:alices_car_rental_use_case_1}.

\begin{lstlisting}[
style=kit-cm,
caption={Alice's Version of Use Case 1},
label={lst:alices_car_rental_use_case_1},
]
1. Alice calls the application CarRental.
2. Alice enters 01.01.00 as a start date and 02.02.00 as the end date as the rental period
3. The system shows a list of the cars that are available at the selected rental period
4. Alice selects a VW ID.2
5. The system shows a rental confirmation and stores the rental

Alternative flows:
3a. The VW ID.2 is not available at the selected time interval
    3a1. The system shows a message that the VW ID.2 is not available at the selected time interval and the flow continues from step 2 or terminates
4a. Alice does not want to rent the VW ID.2
    4a1. The flow terminates
\end{lstlisting}

\label{cha:design_of_data_and_functionality}

\section{Design of the Data and the Functionality}

\subsection*{Description of the Initial Entity Diagram}
The entity diagram consists of the following entities:
\begin{itemize}
    \item Customer entity
    \item Rental entity
    \item Car entity
    \item Date value object
\end{itemize}
A customer has zero ore more rentals.
Each rental has a date value object and belongs to at most one car.

\subsection*{Adding the Attributes}
The extended entity diagram is shown in figure \ref{fig:extendedEntityDiagram}.
The following attributes are added to the entities:
\begin{itemize}
    \item Customer entity
    \begin{itemize}
        \item id: contains the customer id and identifies the customer
        \item name: contains the name of the customer
        \item creditCard: true if the customer has a valid credit card, false otherwise
        \item rentals: list of rental objects that belong to the customer. The list can be empty if no cars are rented.
    \end{itemize}
    \item Rental entity
    \begin{itemize}
        \item id: contains the rental id and identifies the rental
        \item date: link to the date value object that contains the start and end date of the rental
        \item car: link to the car object that is rented. Can be null if no car is assigned to the rental
        \item customer: link to the customer object clearly identifying the customer
    \end{itemize}
    \item Car entity
    \begin{itemize}
        \item id: contains the car id and identifies the car
        \item type: contains the type of the car
        \item rented: true if the car is rented, false otherwise
        \item price: contains the price of the car per day
    \end{itemize}
    \item Date value object
    \begin{itemize}
        \item startDate: contains the start date of the rental
        \item endDate: contains the end date of the rental
    \end{itemize}
\end{itemize}

\begin{figure}[h]
    \centering
    \includegraphics[width=0.6\textwidth]{figures/goLang/carRental/carRental_extendedEntity.png}
    \caption{Extended Entity Diagram}
    \label{fig:extendedEntityDiagram}
\end{figure}

\subsection*{Adding the Functionality as a Method}
The chosen signature of the funcion is \texttt{rentACar(date, rental)}.
The method works as follows:
\begin{enumerate}
    \item The function checks, if the credit card is valid. If not, the function aborts.
    \item The date object is created with the given start and end date and is given to the function.
    \item The given rental object is assigned according to the chosen car / rental.
    \item The algorithm checks, if the car assigned to the rental is available. If not, the function aborts.
    \item The car is marked as rented.
    \item The date object is assigned to the rental.
    \item The rental object is linked to the customer.
    \item All changes are saved to the objects.
\end{enumerate}

\subsection{Excercise CarRentalStructs}
\label{sec:car_rental_structs}
This task introduces the concept of structs and arrays in go.
The goal is to apply the described concept described in the tasks above to the given code.

By creating structs and arrays a general understanding of the concept of storing and accessing data is developed.

\subsubsection*{Add the Attributes to Structs}
According to \cite{W3S-STR}, a struct creates a collection of members of different types into a single object.
The difference to arrays is that the members can be of different types.

A struct can be defined by using the keyword \texttt{type} followed by the name of the struct and the keyword \texttt{struct}.
Then the attributes and their datatypes are defined within curly brackets.
Struct members can be accessed by using the dot operator after the struct name.

Structs can be used to store data in a structured way.
It can also be used for passing and returning multiple values from a function.

The goal of this task is to add the attributes from the entities Car and Rental as specified in \autoref{fig:extendedEntityDiagram} to the structs.

After adding the attributes from the entity diagram to the \hfill \newline \texttt{.../golang/CarRental/CarRentalStructs/CarRentalStructs.go} path and saving
the IDE automatically formats the code and indents it correctly.

The code is specified in \autoref{lst:car_rental_structs}.
\begin{lstlisting}[
style=kit-cm,
language=Golang,
caption={Structs of CarRentalStructs.go after adding the Attributes from the Entity Diagram},
label={lst:car_rental_structs}
]
\\ CarRentalStructs.go
\\ Author: Felix Weik

type Customer struct {
	id         int
	Name       string
	rentals    []Rental
}

type Rental struct {
	id          int
	startDate   Date
    endDate     Date
}

type Car struct {
	id          int
	carType     string
}

type Date struct {
	day         int 
    month       int 
    year        int 
}
\end{lstlisting}

\subsubsection*{Initialize and Print Structs}
Structs are organized in key-value pairs.
The key is the name of the attribute and the value is the value of the attribute.
By using the dot operator, this data can be accessed.
The key to this task is to initialize the structs and print them to the console.

The code for printing the structs is shown in \autoref{lst:car_rental_print_structs}.
The \texttt{fmt.Printf()} function is used to print the structs.
The \texttt{\%+v} is used to print the structs in a human-readable way.
\texttt{\%v} is a formatting verb used to represent values in a default way.
Adding the \texttt{+} flag the data is displayed in a structured way, which is particularly useful for structured data like structs.
The struct-name after the comma defines the struct that will be printed where the placeholder is located.

The result of the initialization and printing of the structs is shown in \autoref{fig:car_rental_structs}.
\begin{figure}[H]
    \centering
    \includegraphics[width=\textwidth]{figures/goLang/carRental/carRental_structs.png}
    \caption{Output of the structs}
    \label{fig:car_rental_structs}
\end{figure}

\begin{lstlisting}[
style=kit-cm,
language=Golang,
caption={Printing the Structs},
label={lst:car_rental_print_structs}
]
// CarRentalStructs.go

fmt.Printf("Customer: %+v\n", customer1)
fmt.Printf("Rental: %+v\n", rental)
fmt.Printf("Car: %+v\n", car)
\end{lstlisting}

\subsubsection*{Create an Array of Rentals}
The function works as follows:
\begin{enumerate}
    \item The array \texttt{cartypes} holds the string of five different car types
    \item The function \texttt{createRentals(id, date, car)} returns five rentals that are appended to the array
    \item Via \texttt{fmt.Println(rentals)} the array is printed into the console
\end{enumerate}

The result is shown in \autoref{fig:car_rental_array_five_rentals}.

The initialization process of an empty array containing five cars in go looks as follows: \texttt{var cars [5]Car}
Alternatively the array can be created already initialized with \texttt{var cartypes = [5]string\{"Type1", ..., "Type5"\}}
This notation creates an array of 5 strings representing cartypes.

A for loop in go ranges from an integer value, usually called "i", to an upper border.
The value will usually be incremented by one until the upper border is reached. 
The code within the loop will be executed until it reaches the upper border.
A correct implementation for five iterations is shown in \autoref{lst:for_loop_init}.

\begin{lstlisting}[
style=kit-cm,
language=Golang,
caption={Initialization of a For-Loop iterating five Times},
label={lst:for_loop_init},
]
for i := 0; i < 5; i++ {
    // Run some code
}
\end{lstlisting}

\begin{figure}[H]
\centering
\includegraphics[width=\textwidth]{figures/goLang/carRental/carRental_arrayFiveRentals.png}
\caption{Output of the Array of Rentals}
\label{fig:car_rental_array_five_rentals}
\end{figure}

\subsection{Excercise CarRentalTests}
\label{sec:car_rental_tests}
This task introduces the concept of testing in Go.
It also introduces the Open-Constraint-Language (OCL) and how to implement it in Go.

A listing is given containing the OCL constraints.
The goal of this task is to analyze these constraints, check their implementation in go, run the tests and add additional test cases.

\subsubsection*{Analyze OCL Constraints}
In the given task four invariants are described.
These invariants stand under the context of the date object.
Therefore the invariants are implemented in the date object.

These invariants are implemented in \texttt{./CarRentalTests/CarRentalTests.go} \hfill \linebreak and executed in \texttt{./CarRentalStructs/CarRentalStructs.go}.

The first invariant checks if the year is greater than or equal to 2000.
The second invariant checks if the month is less than 13.
The third invariant checks if the month is greater than or equal to 1.
The fourth invariant checks if the number of days in the month is valid.

The following enumeration depicts the lines of code where the invariants are implemented.
\begin{enumerate}
    \item \texttt{self.Date.Year >= 2000}: CarRentalTests.go line 20-22
    \item \texttt{self.Date.Month < 13}: CarRentalTests.go line 24-26
    \item \texttt{self.Date.Moth >= 1}: CarRentalTests.go line 24-26
    \item \texttt{self.ValidateNumberOfDaysInMonth() == True}: CarRentalTests.go line 32
\end{enumerate}

\subsubsection*{Run Test}
After running the test function an error occurs. 
The test, therefore, does not pass.
The error is shown in \autoref{fig:car_rental_test_error}.

This error is caused due to a false implementation in line 20 of \hfill \linebreak \texttt{./CarRentalTests/CarRentalTests.go}
In the first test, it checks if the year is greater than 2000, yet for correct execution, it needs to be greater than or equal to 2000.

\begin{figure}[H]
    \centering
    \includegraphics[width=0.8\textwidth]{figures/goLang/carRental/carRental_dateTestError.png}
    \caption{Error of the Test}
    \label{fig:car_rental_test_error}
\end{figure}

\subsubsection*{Correct Code}
As mentioned in the subsection above, line 20 is not implemented correctly.
By changing line 20 from \texttt{if !(d.Year \> 2000)} to \texttt{if !(d.Year \>\= 2000)} the code will execute correctly and the test passes.
The working tree of the changes is shown in \autoref{fig:car_rental_test_working_tree}.

\begin{figure}[H]
    \centering
    \includegraphics[width=\textwidth]{figures/goLang/carRental/carRental_dateTestWorkingTree.png}
    \caption{Working Tree of the Changes}
    \label{fig:car_rental_test_working_tree}
\end{figure}

\subsubsection*{Invalid Test Case}
Implementing an invalid test case the following code is added to the test function:
\begin{lstlisting}[
style=kit-cm,
language=Golang,
caption={Invalid Test Case},
label={lst:invalid_test_case}
]
name:   "InvalidTestCase1",
fields: fields{Day: 12, Month: 13, Year: 2000},
want:   false,  
\end{lstlisting}

As shown in \autoref{lst:invalid_test_case} the month is set to 13, which is invalid.
Therefore the test should return false.
The test, however, runs successfully due to the want value set to false.


%==============================================================================

\section{Advanced Go Program CarRentalCLI}
\label{sec:advanced_go_program_car_rental_cli}
This section introduces advanced programming concepts in Golang.

The actor needs to input data into the CarRental program to interact with it.
Therefore, CarRentalCLI implements a simple command line interface (CLI) for the CarRental program.

The following tasks will introduce the application by analysing the software and micro architecture.
After understanding the program, commands and mappers will be added.
After that, the program will be tested according to a list of OCL constraints.

In the end of this section, further use cases and functionality will be implemented as part of a challenge.

\subsection{Excercise GettingStartedWithCarRentalCLI}
\label{sec:exercise_getting_started_with_car_rental_cli}
\subsubsection*{Analyze the CLI Parameters}
After analyzing the use cases presented in section 3.1 of the goLang tasks in table 4.1, the parameters displayed in figure \ref{fig:car_rental_cli_parameters} are needed.

\begin{figure}[H]
    \centering
    \includegraphics[width=\textwidth]{figures/goLang/carRental/carRental_CLIParameters.png}
    \caption{CLI Parameters}
    \label{fig:car_rental_cli_parameters}
\end{figure}

\subsubsection*{Describe the Software Architecture}
The given task starts with three different software components:
\begin{itemize}
    \item Presentation Layer: CarRentalCLI
    \item Application Logic Layer: CarRentalOperations
    \item Infrastructure Layer: CarRentalRepository
\end{itemize}
\subsubsection*{Presentation Layer: CarRentalCLI}
\begin{itemize}
    \item Functionality: This component implements the CLI itself. 
          It creates a new CLI object, runs it, and exits safely if an error occurs.
          Furthermore, it implements the CLI commands with the list of according functions a command calls.
    \item Interfaces: The CLI component implements the CarRentalOperationsInterface
    \item Dependencies: This layer is dependent from the applications logic layer, importing and implementing functions from the logic layer.
    \item Reusability: The layer is reusable since it only implements CLI and commands.
          By providing a different set of function the CLI implements, one can reuse the CLI in a different context.
    \item Scalability and Maintainability: It can be scaled by adding new commands and functions to the CLI.
          It can be maintained by fixing commands and the according functions.
          By separating the CLI into a single layer, the commands can be organized systematically and provide great overview.
\end{itemize}

\subsubsection*{Application Logic Layer: CarRentalOperations}
\begin{itemize}
    \item Functionality: This component implements the business logic of the application.
          It provides functions for the CLI to call, which then call the according functions from the infrastructure layer.
          It also provides the models of the objects, that are used in the application.
    \item Interfaces: This component implements the CarRentalRepositoryInterface and provides the CarRentalOperationsInterface
    \item Dependencies: This component is dependent from the infrastructure layer and therefore dependent from the layer below.
    \item Reusability: Since this component uses the infrastructure layer, it is not directly reusable.
          It implements a specific set of functions, which cannot be reused in different contexts.
    \item Scalability and Maintainability: It can be scaled by adding new functions to the component.
          It can be maintained by fixing functions.
\end{itemize}

\subsubsection*{Infrastructure Layer: CarRentalRepository}
\begin{itemize}
    \item Functionality: This component implements yaml mappers and entities. 
          This allows for simple communication between the application and the yaml files.
          The yaml mappers provide functions to read and write yaml files.
          The entities provide the structure of the yaml files and the structure of the struct the mapper creates.
    \item Interfaces: This component provides the CarRentalRepositoryInterface
    \item Dependencies: The component is dependent from the provided yaml files, since it reads and writes them.
          In terms of functionality, no functionality imports from other parts of the application are needed since all functions are implemented in the component itself.
    \item Reusability: This component is reusable since it only implements yaml mappers and entities.
          By providing a different set of entities and mappers, one can reuse the component in a different context.
    \item Scalability and Maintainability: By adding new mappers and entities, the functionality can be extended.
          By fixing the mappers and entities, the maintainability can be ensured.
\end{itemize}

\subsubsection*{Analyze the Micro Architecture}

\subsubsection*{Start the Go Program CarRentalCLI}
The output after starting the program successfully is displayed in figure \ref*{fig:car_rental_cli_successful_rental_max}.
The program asks for the id of the customer and the id of the car.
After entering the ids, the program asks for the start and enddate of the rental by asking for the day, month and year.
If the data is entered correctly and the rental is successful, the program prints a success message and exits.
Otherwise, the program asks for the remaining data, prints an error message, the exit-status and exits.

\begin{figure}
      \centering
      \includegraphics[width=0.8\textwidth]{figures/goLang/carRental/carRentalCLI/carRentalCli_SuccessfulRentalMax.png}
      \caption{Successful Rental of a Car for Max}
      \label{fig:car_rental_cli_successful_rental_max}
\end{figure}

\subsection{Excercise AddRegisterAsCustomer}
\label{sec:exercise_add_register_as_customer}
\subsubsection*{Why is the Mapper required?}
The general task of a mapper is to enable communication between the application and the database.
In this case, the database is a yaml file.
Therefore the mapper can read and write to the yaml files and convert the data to the according entities and vice versa.

In the use case of registering a new customer, the mapper is called while creating a new customer.
The yaml mapper receives the customer object from the function and returns the CustomerPersistenceEntity.
This entity is then written to the yaml file.
By writing the entity into the yaml file, the customer is registered and can be used in the application.

\subsubsection*{Register Alice as Customer}
After running the CLI command to register a new customer, the program asks for the ID of the customer.
In this case Alice receives the ID \texttt{Cus5}.
After entering the ID, the program asks for the name of the customer, in this case Alice.
The program exits with a success message and the exit-status as shown in figure \ref{fig:car_rental_cli_register_alice}.

To register a new rental, the programm starts with \texttt{go run . -- rent}.
The program asks for the ID of the rental, here it is \texttt{Rental7}.
The customer ID for Alice is \texttt{Cus5}.
After that the start and enddate is entered.
The program exits with a success message and the exit-status as shown in figure \ref{fig:car_rental_cli_rental_alice_successful}.

\begin{figure}
      \centering
      \includegraphics[width=0.8\textwidth]{figures/goLang/carRental/carRentalCLI/carRentalCLI_RegisterAlice.png}
      \caption{Register Alice as a Customer}
      \label{fig:car_rental_cli_register_alice}
\end{figure}
\begin{figure}
      \centering
      \includegraphics[width=0.8\textwidth]{figures/goLang/carRental/carRentalCLI/carRentalCLI_SuccessfulRentalAlice.png}
      \caption{Successful Rental of a Car for Alice}
      \label{fig:car_rental_cli_rental_alice_successful}
\end{figure}

\subsection{Excercise TestCarRentalCLI}
\label{sec:exercise_test_car_rental_cli}
\subsubsection*{Derive Test Cases}

\subsubsection*{Analyze Test Structure}
% Structure description of the test file and mock-repository
The MockCarRepository acts similar to the actual CarRepository.
It implements the CarRentalRepositoryInterface and provides the according functions.
Furthermore, it also implements the lists of the models \texttt{Rental}, \texttt{Car}, and \texttt{Customer}.
However, there is no constructor and no manipulation of yaml files.
All data-manipulations are done in the memory of the mock-repository.
Therefore the mock-repository only imports the used models.

The test file \texttt{RentACarOperation\_test.go} implements the tests for the \texttt{RentACar} function.
It creates the testint environment by implementing the constructor of the mock-repository, populating it with test data.
After the setup the executable tests are implemented.

% Structure description of the test function TestCarRentalOperations_RentACar
The implementation of the tests happens in the \texttt{TestCarRentalOperations\_RentACar} function.
The \texttt{fields} struct defines the repository used in each test and therefore the test environment.
After that, the \texttt{args} struct defines the arguments needed for the \texttt{RentACar} function.
Now, the test cases are implemented containing the following values:
\begin{itemize}
      \item fields: The repository used in the test as specified above
      \item name: The name of the test
      \item args: The arguments for the \texttt{RentACar} function as specified above
      \item want: The expected return value of the \texttt{RentACar} function, in this case a String 
      \item wantErr: True, if an error is expected, false otherwise. 
            This allows the implementation of negative test cases without the tests failing.
\end{itemize}

% Differences between the mock and the normal repository
The following differences occur between the mock and the normal repository:
\begin{itemize}
      \item The mock repository does not implement a constructor
      \item The mock repository does not manipulate any yaml files, everything is done in the mock repository's lists.
      \item The mock repository only imports the models used in the tests.
      \item The mock repository is only used for testing while the normal repository is used in the application.
      \item The mock repository is located in the same package as the test file (operations), while the normal repository is located in the infrastructure package.
\end{itemize}

However, there are still some similarities:
\begin{itemize}
      \item Both repositories implement the CarRentalRepositoryInterface and therefore provide the same functions.
      \item The provided functionalities are somewhat similar besides the differences mentioned above.
      \item Both repositories are adressed via an operations object representing the repository
\end{itemize}

% Why is a mocked repository required?
Now, why is mocked repository required?
First of all, the mock repository is required to test the functionality of the \texttt{RentACar} function.
This could also be done with the normal repository, however, this would require the manipulation of the yaml files, which arises the following problems:
\begin{itemize}
      \item Test data and actual application data are mixed up
      \item The test data is not reset after the test, which could lead to problems in the next test
      \item The test data needs no be cleaned up after the test, which is not necessary with the mock repository
      \item By creating a certain testing environment, the testing data can manipulated to test the wanted scenarios
\end{itemize}

Further benefits and the actual usage and implementation of both the normal and the mock repository are further explained in \ref*{sec:go_repositories}.

\subsection{Excercise TestCarRentalCLI}
\label{sec:exercise_test_car_rental_cli}
This task is about testing the CarRentalCLI application, to be more specific, the \texttt{RentACar} function.

The \texttt{RentACar} function is located in the \texttt{operations} package in the \hfill \linebreak \texttt{RentACarOperation.go} file.
It takes the rentalID, the customerID, the carID, and start and endDate as arguments and returns a String and an error.

During its execution, it checks the given arguments for validity and checks the pre-condition.
If the pre-condition is met, the function creates a new rental and adds it to the repository.
If one of the checks fails, or an error occurs during runtime, the function returns an error.

After successfully creating the new rental by calling the \texttt{repository.CreateRental()} function, a string indicating the success concatenated with the rental ID is returned.

\texttt{RentACarOperation\_text.go} contains the tests and the according data for the testing environments to test the \texttt{RentACar} function.
The goal of the following tasks is to derive test cases from given OCL constraints, analyze the test structure, add test data, implement and finally run the tests.

\subsubsection*{Derive Test Cases}
For the sake of simplicity, assume that a correct car called \texttt{correctCar} and a customer called \texttt{correctCustomer} are provided.
In the cases where the date is correct assume the date \texttt{startDate} and \texttt{endDate} represent correct dates.
In this example assume the start Date to be 01.01.2000 and the end Date to be 02.02.2000.

\autoref{lstlisting:rental_test_cases} defines positive and negative test cases for the \texttt{NewRental} function.
\begin{lstlisting}[
style=kit-cm,
language=Golang,
caption={Test Cases for the Rental Context},
label={lstlisting:rental_test_cases},
]
// Constraint 1: self.ID -> notEmpty()
NewRental("", startDate, endDate, correctCar, correctCustomer) // negative test case
NewRental("Rental1", startDate, endDate, correctCar, correctCustomer) // positive test case

// Constraint 2: self.startDate -> notEmty()
NewRental("Rental2", date{}, endDate, correctCar, correctCustomer) // negative test case
NewRental("Rental2", startDate, endDate, correctCar, correctCustomer) // positive test case

// Constraint 3: self.endDate -> notEmty()
NewRental("Rental3", startDate, date{}, correctCar, correctCustomer) // negative test case
NewRental("Rental3", startDate, endDate, correctCar, correctCustomer) // positive test case

// Constraint 4: self.startDate < self.endDate
NewRental("Rental4", endDate, startDate, correctCar, correctCustomer) // negative test case
NewRental("Rental4", startDate, endDate, correctCar, correctCustomer) // positive test case

// Constraint 5: self.car -> notEmpty()
NewRental("Rental5", startDate, endDate, car{}, correctCustomer) // negative test case
NewRental("Rental5", startDate, endDate, correctCar, correctCustomer) // positive test case

// Constraint 6: self.customer -> notEmpty()
NewRental("Rental6", startDate, endDate, correctCar, customer{}) // negative test case
NewRental("Rental6", startDate, endDate, correctCar, correctCustomer) // positive test case
\end{lstlisting}
  
This paragraph designs positive and negative test cases for the \texttt{rentACar\(\)} function.
This function takes the rental ID, start and end date, the car, and the customer as arguments.
If the function is executed correctly it will return the new rental.
As specified at the beginning of the section, startDate, endDate, correctCar, and correctCustomer represent the correct parameters.

\begin{lstlisting}[
style=kit-cm,
language=Golang,
caption={Test Cases for the Customer Context},
label={lstlisting:customer_test_cases},
]
// Constraint 1: pre: self.rentalID -> not exists
rentACar("Rental1", startDate, endDate, correctCar, correctCustomer)
rentACar("Rental1", startDate, endDate, correctCar, correctCustomer) // negative test case due to function executing twice
rentACar("Rental2", startDate, endDate, correctCar, correctCustomer) // positive test case

// Constraint 2: and Rental.allInstances() -> 
//  forAll(r: Rental | r.car = self.car implies(self.endDate < r.endDate or r.Endate > self.startDate))
rentACar("Rental3", startDate, endDate, correctCar, correctCustomer)
rentACar("Rental4", startDate', endDate, correctCar, correctCustomer') // negative test case due to overlapping dates for the same car
rentACar("Rental5", startDate', endDate', correctCar, correctCustomer') // positive test case 

// Constraint 3: post: self.rentals -> includes(rental)
rentACar("Rental6", startDate, endDate, correctCar, correctCustomer) // positive test case
rentACar("Rental7", date{}, endDate, correctCar, correctCustomer) // negative test case => due to constraint 2, the element will not be created and therefore will not appear in the list

\end{lstlisting}

\subsubsection*{Analyze Test Structure}
% Structure description of the test file and mock-repository
The MockCarRepository acts similarly to the actual CarRepository.
It implements the CarRentalRepositoryInterface and provides the appropriate functions.
Furthermore, it also implements the lists of the models \texttt{Rental}, \texttt{Car}, and \texttt{Customer}.
However, there is no constructor and no manipulation of yaml files.
All data manipulations are done in the memory of the mock repository.
Therefore the mock-repository only imports the used models.

The test file \texttt{RentACarOperation\_test.go} implements the tests for the \texttt{RentACar} function.
It creates the testing environment by implementing the constructor of the mock repository and populating it with test data.
After the setup, the executable tests are implemented.

% Structure description of the test function TestCarRentalOperations_RentACar
The tests are implemented in the \texttt{TestCarRentalOperations\_RentACar} function.
The \texttt{fields} struct defines the repository used in each test and therefore the test environment.
After that, the \texttt{args} struct defines the arguments needed for the \texttt{RentACar} function.
Now, the test cases are implemented containing the following values:
\begin{itemize}
      \item fields: The repository used in the test as specified above
      \item name: The name of the test
      \item args: The arguments for the \texttt{RentACar} function as specified above
      \item want: The expected return value of the \texttt{RentACar} function, in this case a String 
      \item wantErr: True, if an error is expected, false otherwise. 
            This allows the implementation of negative test cases without the tests failing.
\end{itemize}

% Differences between the mock and the normal repository
The following differences occur between the mock and the normal repository:
\begin{itemize}
      \item The mock repository does not implement a constructor
      \item The mock repository does not manipulate any yaml files, everything is done in the mock repository's lists.
      \item The mock repository only imports the models used in the tests.
      \item The mock repository is only used for testing while the normal repository is used in the application.
      \item The mock repository is located in the same package as the test file (operations), while the normal repository is located in the infrastructure package.
\end{itemize}

However, there are still some similarities:
\begin{itemize}
      \item Both repositories implement the CarRentalRepositoryInterface and therefore provide the same functions.
      \item The provided functionalities are somewhat similar besides the differences mentioned above.
      \item Both repositories are addressed via an operations object representing the repository
\end{itemize}

% Why is a mocked repository required?
Now, why is a mocked repository required?
First of all, the mock repository is required to test the functionality of the \texttt{RentACar} function.
This could also be done with the normal repository, however, this would require the manipulation of the yaml files, which raises the following problems:
\begin{itemize}
      \item Test data and actual application data are mixed up
      \item The test data is not reset after the test, which could lead to problems in the next test
      \item The test data needs to be cleaned up after the test, which is not necessary with the mock repository
      \item By creating a certain testing environment, the testing data can be manipulated to test the wanted scenarios
\end{itemize}

Further benefits and the actual usage and implementation of both the normal and the mock repository are further explained in \autoref{sec:go_repositories}.

As shown all tests pass besides the first test.
This is due to the given testing environment.
The rental is already created therefore an error will occur and the test will fail.

\subsubsection*{Add Test Data}
As mentioned above, \texttt{RentACarOperation\_test.go} implements the tests for the \texttt{RentACar} function.

The function \texttt{SetupMockCarRentalRepository()} implements the constructor of the mock repository.
It takes no arguments and returns a MockCarRentalRepository.

The function creates arrays of customers, cars and rentals.
These arrays are then used to initialize the mock repository.
The mock repository is then returned.

To populate an array, it needs to be filled with the according structs as specified in the model package.
The structs are initialized in the array.
An example of the initialization of the customer's array is shown in \autoref{lstlisting:customers_array}.

\begin{lstlisting}[
style=kit-cm,
language=Golang,
caption={Initialization Example of the Customers Array},
label={lstlisting:customers_array},
]
customers := []model.Customer{{
      ID:   "Cus1",
      Name: "Max",
}, {
      ID:   "Cus2",
      Name: "Bob",
}, {
      ID:   "Cus3",
      Name: "Simon",
}}
\end{lstlisting}

If a value is referenced in another struct, the reference is set to the according struct in the different array.
During the initialization of the rentals array, this occurs, for example, as shown in \autoref{lstlisting:rentals_array}.

\begin{lstlisting}[
style=kit-cm,
language=Golang,
caption={Code Snippet of Initializing the Rentals Array},
label={lstlisting:rentals_array},
]
rentals := []model.Rental{{
ID: "Rental1",
StartDate: model.Date{
      Day:   1,
      Month: 3,
      Year:  2000,
},
EndDate: model.Date{
      Day:   15,
      Month: 3,
      Year:  2000,
},
Car:      cars[0],
Customer: customers[0],
}, {
ID: "Rental2",
StartDate: model.Date{
      Day:   3,
      Month: 2,
      Year:  2000,
},
EndDate: model.Date{
      Day:   5,
      Month: 2,
      Year:  2000,
},
Car:      cars[1],
Customer: customers[1],
},
...
}
\end{lstlisting}

\subsubsection*{Implement Test Cases}
The tests are implemented in the \texttt{TestCarRentalOperations\_RentACar(t \*testing.T)} function.
The function takes a testing object as an argument and returns nothing.

The function starts by defining the fields and args structs.
The field struct defines the repository used in each test and therefore the test environment.
The args struct defines the arguments needed for the \texttt{RentACar} function - The rental ID, the customerID, the carID, the start and end date.

After that, the test cases are implemented in the \texttt{tests} array.
A test case consists of the test name, the fields, args, the expected return string and a boolean indicating if an error is expected, called \texttt{wantErr}.
Just like in \autoref{lstlisting:customers_array} the arrays are initialized with the according structs.

This task is about implementing the test cases specified in \autoref{lstlisting:rental_test_cases} in the \hfill \linebreak \texttt{TestCarRentalOperations\_RentACar}'s \texttt{tests} array.
A code snippet of two fully implemented test cases is shown in \autoref{lstlisting:test_case_implementation}.

\texttt{NegativeRentalsOverlap} is an example of a negative test case.
A similar rental renting the same car has already been created.
Since the dates of both rentals overlap, a new rental cannot be created.

\begin{lstlisting}[
      style=kit-cm,
      language=Golang,
      caption={Code Snippet of the Test Case Implementation},
      label={lstlisting:test_case_implementation},
]
tests := []struct {
      name    string
      fields  fields
      args    args
      want    string
      wantErr bool
}{
      
		{
			name:   "NegativeRentalsOverlap",
			fields: fields{repository: SetupMockCarRentalRepository()},
			args: args{
				rentalID:   "Rental6",
				customerID: "Cus4",
				carID:      "Car5",
				startDate: model.Date{
					Day:   6,
					Month: 2,
					Year:  2000,
				},
				endDate: model.Date{
					Day:   12,
					Month: 3,
					Year:  2000,
				},
			},
			want:    "",
			wantErr: true,
		},
      ... //Further test cases
}
\end{lstlisting}

\subsection*{Run Tests}
Tests can be started by using Visual Studio Code's graphical features.
By clicking on the starting button next to the function definition of \texttt{TestCarRentalOperations\_RentACar()}, the function is executed as a test.

VSC offers a "Test Result" bar in the terminal panel, visualizing the tests executed and also specifying the start time, runtime and time of completion.
This output is shown in \autoref{fig:carRentalCLI_testingConsoleOutput}.

Right next to this bar, a graphical representation of the tests is shown.
This is shown in \autoref{fig:carRentalCLI_testingGraphicalOutput}.
The green checkmarks indicate successful tests, while the red crosses indicate failed tests.
Single tests can be rerun if wanted by clicking on the according test.
This allows the developer to quickly rerun tests offering a more efficient debugging process.

As mentioned above \autoref{fig:carRentalCLI_testingConsoleOutput} and \autoref{fig:carRentalCLI_testingGraphicalOutput} show the testing results of the tests specified in the previous subtasks.

The first test fails, since it is designed to be a positive test case, however, the rental is already created.
This leads to an error during execution and the test fails due to the \texttt{wantErr} attribute being set to \texttt{false}.

The other tests succeed since they are designed to be negative test cases therefore expecting an error during execution.

\begin{figure}
      \centering
      \includegraphics[width=0.8\textwidth]{figures/goLang/carRental/carRentalCLI/carRentalCLI_testingConsoleOutput.png}
      \caption{Console Output of the RentACar Tests}
      \label{fig:carRentalCLI_testingConsoleOutput}
\end{figure}
\begin{figure}
      \centering
      \includegraphics[width=0.8\textwidth]{figures/goLang/carRental/carRentalCLI/carRentalCLI_testingGraphicalOutput.png}
      \caption{Graphical Output of the RentACar Tests}
      \label{fig:carRentalCLI_testingGraphicalOutput}
\end{figure}




\section{Challenge: Final Use Cases and Further Functionality}
\label{sec:challenge_final_use_cases_and_further_functionality}

\subsection{Challenge AddMissingFunctionalitiesToCarRentalCLI}
\label{subsec:challenge_addmissingfunctionalitiestocarrentalcli}

\subsubsection*{Implement Use Cases}
Two use cases are missing in the CarRentalCLI program.

The first use case is to cancel an already existing rental.

The implementation process is quite similar to the already implemented use cases.
\texttt{CarRentalCLI.go} holds the CLI commands.
By adding "cancel" to the commands array, the command can be used in the CLI.
The command calls the \texttt{CancelRentalAction} function, taking the operations as input.

\texttt{CancelRentalAction} prints "Cancel a rental" and collects the rentalID from the user.
The rental is then canceled by calling \texttt{CancelRental}, taking the rentalID as input.

\texttt{CancelRental} verifies the rentalID by checking if it exists. \hfill \linebreak
It then calls \texttt{DeleteRental(rentalID)} to delete the rental.
The \texttt{DeleteRental} function located in the repository removes the rental from the repository, then saves the data to the yaml files.

If everything runs successfully, a success message is printed to the user containing the removed rental ID.
If an error happens during runtime, the application exits safely.

To test the function, a new file called \texttt{CancelRentalOperation\_test.go} is created.
It uses the already implemented \texttt{SetupMockCarRentalRepository} function from \hfill \linebreak \texttt{CarRentalRepository\_test.go} to create a mock repository.

The function running the tests is similar to the already implemented test function from \texttt{CarRentalRepository\_test.go} besides the test cases.
The test cases only contain the rental ID as a parameter, since it is the only input needed to delete a rental.

Two test cases are implemented:
The first test case is nonconflicting and deletes the already existing first rental.
The second test case is built to fail by wanting to delete a non-existing rental.

Both tests succeed as shown in \autoref{fig:carRentalCLI_challenge_successfulCancellationOfRentalTest}.

\begin{figure}
    \centering
    \includegraphics[width=0.8\textwidth]{figures/goLang/carRental/carRentalCLI/challenge/carRentalCLI_challenge_successfulCancellationOfRentalTest.png}
    \caption{Successful Tests of the Newly Implemented CancelRental Function}
    \label{fig:carRentalCLI_challenge_successfulCancellationOfRentalTest}
\end{figure}


%==============================================================================