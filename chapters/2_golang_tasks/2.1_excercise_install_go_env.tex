\label{sec:exercise_install_go_env}

\subsection{Exercise M2GoEnvironmentInstallation}
\label{sec:exercise_m2go_environment_installation}

\subsubsection*{Environment Description}
The hardware used is a Windows Surface Studio with an 11th Gen Intel Core i5-11300H, 16 GB of RAM, and 256 GB of SSD storage.
The operating system is Windows 11 64-bit. However, the installation is carried out on WSL 2 running Ubuntu 20.04.6 LTS installed on the machine.

\subsubsection*{Installation Steps}
WSL2 (Ubuntu 20.04.6 LTS), VSCode (1.84.2) and Git (2.25.1) are already installed on the machine. 
Therefore no further installation steps are required for these tools.
The correct environment-installation on the author's machine can be seen in \autoref{fig:screendump_installation}.

Carrying out the installation of Go, version go1.21.4 is installed.
To install Go on the machine, the following steps are required:
\begin{enumerate}
    \item Download the latest version of Go from \url{https://go.dev/dl/go1.21.4.linux-386.tar.gz} by using  \texttt{wget} in the terminal.
    \item Extract the downloaded archive to \texttt{/usr/local} via \texttt{sudo tar -C /usr/local -xzf go1.21.3.linux-386.tar.gz}
    \item Under \texttt{/etc/profile} add the following lines to the bottom of the file: 
    \begin{itemize}
        \item \texttt{export PATH=\$PATH:/usr/local/go/bin}
        \item export \texttt{GOPATH=\$HOME/go}
    \end{itemize}
    \item Now reload the terminal or force the update by running \texttt{source \$HOME/.profile}
    \item Verify the installation by running \texttt{go version}
\end{enumerate}

\begin{figure}[H]
	\centering
	\includegraphics[width=0.5\textwidth]{figures/goLang/installation_screendump.png}
	\caption{Screendump showing the correct Installation of the Environment}
	\label{fig:screendump_installation}
\end{figure}

Further details on the installation and especially the path variables can be found in \autoref{sec:go_environment_variables} on path variables in Ubuntu.