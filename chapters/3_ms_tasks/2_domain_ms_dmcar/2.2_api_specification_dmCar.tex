\subsection{OpenAPIDMCar}
\subsubsection{Markup Language}
\autoref*{lst:openapi-spec} shows the generic structure of an OpenAPI specification.
This listing also can be found in the tasks sheet \cite{CM-T-DMC}.

The used language is YAML (YAML Ain't Markup Language).
An alternative is JSON (JavaScript Object Notation).
Yet, there are some advantages of YAML over JSON in this context.

YAML is a superset of JSON, which means that every JSON file is also a valid YAML file.
However, YAML is more readable and easier to write than JSON.
JSON requires a more complex syntax, which makes it more error-prone.
The specification only needs to be readable for humans, no machine needs to parse it.
Therefore, JSON is more suitable for data interchange, while YAML is more suitable for configuration files.

\begin{lstlisting}[
    style=kit-cm,
    caption={Generic OpenAPI Specification Structure},
    label={lst:openapi-spec},
]
openapi : 3.0.1
info :
    title : API Specification < Microservice Name >
    description : < Description >
    version : 1.0.0
tags :
paths :
components :
    schemas :
\end{lstlisting}