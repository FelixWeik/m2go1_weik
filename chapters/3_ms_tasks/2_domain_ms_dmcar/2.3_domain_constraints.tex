\subsection{Excercise DomainConstraints}
\subsubsection*{Analyze the Domain Constraints}
The constraints describe invariants, post-, and pre-conditions of a certain context that have to be fulfilled.
This happens in two different contexts: \texttt{Car} and the \texttt{addCar()} method of the \texttt{CarCollection}.

\paragraph*{Context Car:}
Three invariants have to be fulfilled for a car to be valid.

First, the VIN has to match a certain regular expression.
This regular expression is defined in the constraint.
The first 13 characters have to be letters reaching from A to H, J to N, R to Z, P, or digits from 0 to 9.
The last four characters must be digits from 0 to 9.
Therefore each VIN is exactly 17 characters long and matches the regular expression.

Second, the brand of the car must not be empty.
Finally, the model of the car must not be empty.

If all invariants are met, the car is valid.

\paragraph*{Context CarCollection::addCar(car: Car):}
This constraint focuses on the \texttt{addCar()} method of the \texttt{CarCollection}.
Here, pre-, and post-conditions are defined.

The pre-condition is executed on all the collection's cars.
This is done by using the \texttt{forAll()} expression.
In each iteration, two cars are compared, to be precise their VINs.
The VINs must not be equal.
By using the \texttt{forAll} expression, all cars are compared to each other.
Therefore, every car in the collection must have a unique VIN for the pre-condition to be true.

The post-condition is executed on the car that is added to the collection, the input of the \texttt{addCar()} method.
It specifies that the car exists in the collection after executing the method.
This is done by using the \texttt{exists()} expression.
Therefore, if the post-condition is true, the car has successfully been added to the collection.

If these conditions are met, the \texttt{addCar()} method executes correctly and the car is successfully added to the collection.

\subsubsection*{Derive Test Cases}

\subsubsection*{Add Constraints to the OpenAPI Specification}