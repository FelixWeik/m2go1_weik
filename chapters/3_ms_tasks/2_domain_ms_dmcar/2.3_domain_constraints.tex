\subsection{Excercise DomainConstraints}
\subsubsection*{Analyze the Domain Constraints}
The constraints describe invariants, post-, and pre-conditions of a certain context that have to be fulfilled.
This happens in two different contexts: \texttt{Car} and the \texttt{addCar()} method of the \texttt{CarCollection}.
The constraints can be found in \autoref{lst:domain_constraints} which is a copy of Listing 2.2 from the task sheet \cite{CM-T-DMC}.

\paragraph*{Context Car:}
Three invariants have to be fulfilled for a car to be valid.

First, the VIN has to match a certain regular expression.
This regular expression is defined in the constraint.
The first 13 characters have to be letters reaching from A to H, J to N, R to Z, P, or digits from 0 to 9.
The last four characters must be digits from 0 to 9.
Therefore each VIN is exactly 17 characters long and matches the regular expression.

The succeeding two constraints demand the \texttt{brand} and \texttt{model} of the car not to be empty.

If all invariants are met, the car is valid.

\paragraph*{Context CarCollection::addCar(car: Car):}
This constraint focuses on the \texttt{addCar()} method of the \texttt{CarCollection}.
Here, pre-, and post-conditions are defined.

The pre-condition is executed on all the collection's cars.
This is done by using the \texttt{forAll()} expression.
In each iteration, two cars are compared, to be precise their VINs.
The VINs must not be equal.
By using the \texttt{forAll} expression, all cars are compared to each other.
Therefore, every car in the collection must have a unique VIN for the pre-condition to be true.

The post-condition is executed on the car that is added to the collection, the input of the \texttt{addCar()} method.
It specifies that the car exists in the collection after executing the method.
This is done by using the \texttt{exists()} expression.
Therefore, if the post-condition is true, the car has successfully been added to the collection.

If these conditions are met, the \texttt{addCar()} method executes correctly and the car is successfully added to the collection.

\subsubsection*{Derive Test Cases}
This task is about deriving negative and positive test cases from the constraints in \autoref{lst:domain_constraints}.

\paragraph*{Context Car:}
Three negative and positive test cases can be derived from the given constraints.
The results are shown in code-form in \autoref{lst:test_cases_car}.

The \texttt{newCar()} constructor is assumed to be used to create a new car.

\begin{lstlisting}[
    float=h,
    style=kit-cm,
    caption={Test Cases for the Context Car},
    label={lst:test_cases_car}, 
]
/*
A car is created by providing a VIN, a brand, and a model.
Therefore, a car is created by using the newCar(vin, brand, model) constructor.
*/

// Positive Test Case 1: VIN matches the regular expression
correctVin1 := "JH4DB1561NS000565"
correctCar1 := newCar(correctVin1, "VW", "ID2")

// Negative Test Case 1: VIN does not match the regular expression
incorrectVin1 := "JH4DB1561NS00056"
incorrectCar1 := newCar(incorrectVin1, "VW", "ID2") // VIN is too short, does not match the regular expression

// Positive Test Case 2: Brand is not empty
correctVin2 := "JH4DB1561NS000566"
correctCar2 := newCar(correctVin2, "VW", "ID2")

// Negative Test Case 2: Brand is empty
incorrectCar2 := newCar(correctVin2, "", "ID2") // brand is empty, does not meet the second constraint

// Positive Test Case 3: Model is not empty
correctVin3 := "JH4DB1561NS000567"
correctCar3 := newCar(correctVin3, "VW", "ID2")

// Negative Test Case 3: Model is empty
incorrectCar3 := newCar(correctVin3, "VW", "") // model is empty, does not meet the third constraint
\end{lstlisting}

\paragraph*{Context addCar(car: Car):}
Two three negative and positive test cases can be derived from the given constraints.

First, the pre-condition is tested.
A positive test case occurs if the VIN of the car that is added to the collection is unique and therefore does not match any of the already stored cars in the collection.
A negative test case occurs if the VIN of the car that is added to the collection is not unique and therefore matches the VIN of at least one of the already stored cars in the collection.

Second, the post-condition is tested.
A positive test case occurs if the car that is added to the collection exists in the collection after the \texttt{addCar()} method has been executed.
A negative test case occurs if the car that is added to the collection does not exist in the collection after the \texttt{addCar()} method has been executed.

A last set of test cases can be derived by using both the pre-, and post-condition.
A positive test case occurs if the car that is added to the collection is unique and exists in the collection after the \texttt{addCar()} method has been executed.
A negative test case occurs if the car that is added to the collection is not unique and does not exist in the collection after the \texttt{addCar()} method has been executed.

\subsubsection*{Add Constraints to the OpenAPI Specification}
Including the constraints in the API has several advantages.

OCL constraints are valid during the whole object lifecycle.
Therefore including them in the API specification ensures that the constraints are met during the whole lifecycle of the object.
This makes the API accept only valid objects, therefore ensuring type safety and the integrity of the data.

% //TODO: Was könnte man hier noch machen?
% //TODO: Den Spaß in die API Specification einbauen

\begin{lstlisting}[
    float=h,
    style=kit-cm,
    caption={Domain Constraints as found in the Task Sheet},
    label={lst:domain_constraints}, 
]
context Car inv:
    self.vin.matches(^[A-HJ-NPR-Z0-9]{13}[0-9]{4}$)
    and self.brand -> notEmpty()
    and self.model -> notEmpty()

context CarCollection::addCar(car: Car):
    pre: allInstances() -> forAll(car1, car2: Car | car1.vin <> car2.vin)
    post: self.getCar -> exists(car)
\end{lstlisting}