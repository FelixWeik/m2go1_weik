\subsection{Excercise DMCarAMRentalManagement}
This task gives a first introduction to the application microservice AM-RentalManagement by comparing it to the domain microservice DM-Car.
First, the task names three matching properties of both microservices.
Then, the task names three properties in which both microservices do not match.
Finally, the UME artifacts containing relations between both microservices are shown.

\subsubsection*{Matching Properties}
This task names three properties in which DM-Car and AM-RentalManagement match:
\begin{itemize}
    \item Both applications are written in Go
    \item Both applications are deployed as Docker containers
    \item Both applications are microservices
\end{itemize}
\subsubsection*{Not Matching Properties}
This task names three properties in which DM-Car and AM-RentalManagement do not match:
\begin{itemize}
    \item DM-Car is a domain microservice, implementing the domain model, while AM-RentalManagement is an application microservice, implementing the application logic.
    \item DM-Car is a stateless microservice, while AM-RentalManagement is a stateful microservice %TODO ist das so?
    \item DM-Car is a microservice that is not exposed to the outside world, while AM-RentalManagement is a microservice that is exposed to the outside world.
\end{itemize}

\subsubsection*{Modeling of the Microservice Relation}
The UME artifacts of DM-Car and AM-RentalManagement contain some relations.
Both microservices are modeled with an API diagram, which shows the general structure and relations of both microservices.
