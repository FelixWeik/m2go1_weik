\subsection{Excercise DMCarAMRentalManagement}
This task gives a first introduction to the application microservice AM-RentalManagement by comparing it to the domain microservice DM-Car.
First, the task names three matching properties of both microservices.
Then, the task names three properties in which both microservices do not match.
Finally, the UME artifacts containing relations between both microservices are shown.

\subsubsection*{Matching Properties}
DM-Car and AM-RentalManagement share three common properties:
\begin{itemize}
    \item Both applications are written in Go.
    \item Both applications follow the UME approach.
    \item Both applications use an SQL database.
\end{itemize}
\subsubsection*{Not Matching Properties}
This task names three properties in which DM-Car and AM-RentalManagement do not match:
\begin{itemize}
    \item DM-Car is a domain microservice implemented in the domain logic layer, while AM-RentalManagement is an application microservice implemented in the application logic layer.
    \item DM-Car defines a ReST API, while AM-RentalManagement defines a gRPC API.
    \item DM-Car implements CRUD operations for the \texttt{Car} domain entity, while \hfill \linebreak AM-RentalManagement focuses on the management of rentals.
\end{itemize}

\subsubsection*{Modeling of the Microservice Relation}
The software architecture illustrated in the component diagram illustrates the relation between both microservices.
An API relationship is shown there.
The system architecture shows the relation between both microservices in deployment.
The orchestration diagram shows how both microservices interact in certain use cases.