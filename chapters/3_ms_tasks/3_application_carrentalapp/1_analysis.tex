\subsection{Excercise AnalysisArtifactsAMMangementV1.0}
TODO: Add introduction
\subsubsection*{Requirements Analysis of CarRentalCLI}
Taking a closer look at the functions of RentalManagementV1.0, the functions of the CarRentalCLI can be spotted.
\texttt{rentCar}, and \texttt{cancelRental} have already been implemented in CarRentalCLI.
To implement these functions correctly the OCL constraints used to analyze the CarRentalCLI can be reused.
Therefore the use cases for \texttt{rentCar} and \texttt{cancelRental} can be reused as well.
Also the \texttt{car} model, although using the domain entity \texttt{Car} from the domain microservice, is the same model as described in CarRentalCLI's requirements analysis.
% TODO: Herausfinden was andere dazu geschrieben haben
\subsubsection*{Terms Used V1.0}
A type-name pair is used to describe the terms used in the requirements analysis.
Each type-name pair is described in a table, in this case, it is the table \texttt{Terms Used V1.0} from the \texttt{0.DocCarRentalAppV1} repository found at \url{https://gitlab.kit.edu/kit/cm/teaching/carrentalapp/0.doccarrentalappv1}.

% TODO: Herausfinden was andere dazu geschrieben haben und was ein type-name pair ist
\subsubsection*{Analysis Artifacts and Application Microservice}
% Aus welchen Teilen ist ... hergeleitet
% AM RentalManagementV1.0
In the UME approach, an application microservice represents the application's logical layer.
It therefore implements the business logic, which is not part of the domain model and therefore application specific.
As mentioned in the task, \texttt{AMRentalManagementV1.0} represents the application logic layer of the \texttt{CarRentalApp} application.
The analysis artifacts are derived from the already analyzed applications \texttt{CarRentalCLI} and \texttt{CarRental}.
These artifacts are the use cases and the OCL constraints.

% Operations von AM-RentalManagementV1.0
The application's operations are derived from the use cases of the \texttt{CarRentalCLI} and \texttt{CarRental} applications.
To be more precise, of the use cases \texttt{Alice's Car Rental}, the use case \texttt{Rent a Car} and the according use case diagram shown in \autoref*{fig:car_rental_use_case_diagram}.
By taking a closer look at the API diagram of \texttt{AM-RentalManagementV1.0}, the CLI commands used in \texttt{CarRentalCLI} are partly represented as operations.
In this case, it is the CLI command for renting a car and canceling a rental.
Further functionality is dependent on the chosen entities and their attributes.
Due to the entities' functions being mostly getter and setter methods, besides the already mentioned functionality, the operations are not derived from already existing operations.