\subsection{Excercise APIDIagramAMRentalManagementV1.0}
TODO Add introduction
\subsubsection*{Derivation of the API Diagram}
An API diagram models the relevant entities depending on the type of API, in this case, \hfill \linebreak \texttt{RentalManagementV1.0} is an application microservice.
In an AM API diagram, relations between the application entities and the entities from other API diagrams occur.
A copy of the API diagram can be found on \url{https://gitlab.kit.edu/kit/cm/teaching/carrentalapp/0.doccarrentalappv1/-/blob/main/figures/ad_am-rental_management_v1.0.png?ref_type=heads} is shown in \autoref*{fig:ad_am-rental_management_v1.0}.

% Which artifacts are used to derive the API diagram
Since \texttt{RentalManagementV1.0} interacts with the \texttt{DM-CarV1.0} domain microservice, the DM API diagram is one artifact used to derive the AM API diagram.
Using the DM API diagram, the application entity \texttt{Car} is modeled.
Next, the entity diagram of \texttt{CarRental} is used to derive the application entities, their functions and their relations.
This entity diagram is shown in \autoref{fig:extendedEntityDiagram}
Yet, not all functions can be derived by \texttt{CarRental} alone.
The Use Cases of \texttt{CarRental} are used to derive functions too.
Advancing in this thesis, these Use Cases are also used to derive the CLI commands for \texttt{CarRentalCLI}.
Despite not all commands being implemented in this microservice, the functions for renting and canceling a rental are implemented, being specified by the Use Cases and the according CLI commands.

% How are the entities and methods established
The application entities \texttt{Customer} and \texttt{Rental} are derived from the entity diagram \autoref{fig:extendedEntityDiagram}.
Both entities keep the same attributes and relations as in the entity diagram.
The entity \texttt{Car} is derived from the DM API diagram, yet the relation between \texttt{Car} and \texttt{Rental} stays the same as in the entity diagram.
Last, the collection \texttt{Rentals} is newly created.
It contains a list of rentals and the \texttt{listAvailableCars} function returning a list of available cars.
Therefore, it is derived from the general structure and functionality of \texttt{DM-CarV1.0}'s \texttt{Cars} collection.

\begin{figure}
    \centering
    \includegraphics[width=0.8\textwidth]{figures/microservices/rentalManagement/ms_rentalManagement_apiDiagram.png}
    \caption{API Diagram of \texttt{AM-RentalManagementV1.0}}
    \label{fig:ad_am-rental_management_v1.0}
\end{figure}

\subsubsection*{Versioning of the API Diagram}

\subsubsection*{Relationship between Rental and Car}
