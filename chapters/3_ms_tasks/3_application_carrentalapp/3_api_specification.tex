\subsection{Excercise gRPCAMRentalManagementV1.0}
TODO: add an introduction

\subsubsection*{Specification Language}
% Describe which API specification language is used for gRPC
gRPC uses the protocol buffer language to define the API specification \cite{PRO-DOC}.
The protocol buffer language is platform-independent and language-agnostic.
Using this language, data structures and services can be defined.

The API specification is defined in a \texttt{.proto} file serving as the groundwork for the implementation of the client- and server-implementation.
The \texttt{.proto} file is compiled by a protobuf compiler to generate the client- and server-side code.
This code then defines the data structures and service stubs of the \texttt{.proto} file.

In the \texttt{.proto} file, messages and services can be defined.
Messages are used to define data structures.
They rank optional and required fields, specifying how the data will be serialized.
After compiling a message using the protobuf compiler, the message can be used as a class in the target language.
This class can then be used to create instances of the message, serializing and deserializing them.
The serialized data can then be used for transmission, persisting or storing.
A service defines remote procedures.
They hold one or several functions, specifying the parameters and return values of the function.
The logic of the function can be directly implemented in the service definition, or it can be implemented in a separate class.
The service can then be used to create a server stub and a client stub.
Therewith, the service can be used to implement a server and a client, enabling communication by specifying the remote procedure calls.

\subsubsection*{Constructing gRPC Services}

\subsubsection*{Map Enumerations to gRPC Enums}

\subsubsection*{Translate Entities and Value Objects to gRPC Messages}

\subsubsection*{Transfer Parameters for gRPC Function}