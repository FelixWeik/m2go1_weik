\subsection{Excercise gRPCAMRentalManagementV1.0}
TODO: add an introduction

\subsubsection*{Specification Language}
% Describe which API specification language is used for gRPC
gRPC uses the protocol buffer language to define the API specification.
The protocol buffer language is platform-independent and language-agnostic.
Using this language, data structures and services can be defined.

The API specification is defined in a \texttt{.proto} file serving as the groundwork for the implementation of the client- and server-implementation.
The \texttt{.proto} file is compiled by a protobuf compiler to generate the client- and server-side code.
This code then defines the data structures and service stubs of the \texttt{.proto} file.


% Explain the language in the context of gRPC
\subsubsection*{Constructing gRPC Services}

\subsubsection*{Map Enumerations to gRPC Enums}

\subsubsection*{Translate Entities and Value Objects to gRPC Messages}

\subsubsection*{Transfer Parameters for gRPC Function}