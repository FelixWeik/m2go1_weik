\subsection{Excercise gRPCAMRentalManagementV1.0}
TODO: add an introduction

\subsubsection*{Specification Language}
% Describe which API specification language is used for gRPC
gRPC uses the protocol buffer language to define the API specification \cite{PRO-DOC}.
The protocol buffer language is platform-independent and language-agnostic.
Using this language, data structures and services can be defined.

The API specification is defined in a \texttt{.proto} file serving as the groundwork for the implementation of the client- and server-implementation.
The \texttt{.proto} file is compiled by a protobuf compiler to generate the client- and server-side code.
This code then defines the data structures and service stubs of the \texttt{.proto} file.

In the \texttt{.proto} file, messages and services can be defined.
Messages are used to define data structures.
They rank optional and required fields, specifying how the data will be serialized.
After compiling a message using the protobuf compiler, the message can be used as a class in the target language.
This class can then be used to create instances of the message, serializing and deserializing them.
The serialized data can then be used for transmission, persisting or storing.
A service defines remote procedures.
They hold one or several functions, specifying the parameters and return values of the function.
The logic of the function can be directly implemented in the service definition, or it can be implemented in a separate class.
The service can then be used to create a server stub and a client stub.
Therewith, the service can be used to implement a server and a client, enabling communication by specifying the remote procedure calls.

\subsubsection*{Constructing gRPC Services}
% Which parts of the API diagram are mapped to gRPC services and why?
% How are methods within an application entity transformed into gRPC functions according to the guideline?
The guidelines on deriving gRPC specifications are provided in the \hfill \linebreak \texttt{1.EngineeringKnowledge > 1.UMEPhases} GitLab Repository \cite{CM-G-RPC}.
It defines certain rules on how and what to derive from a given API diagram into the gRPC specification.
First, each entity or collection that houses gRPC functions is mapped to a gRPC service.
The service is named after the entity or collection with the suffix \texttt{Service} or \texttt{CollectionService}.
Also, the gRPC functions are named after the corresponding methods of the entity or collection.
Next, each enumeration is mapped to a gRPC enumeration with each value being translated to a gRPC enumeration value.
Then, each entity is mapped to a gRPC message having the same name and attributes as the entity.
Finally, each input and output parameter of the defined gRPC functions is defined as a single message.
Each message contains the attributes of the input or output parameter.
These messages are named after the corresponding method with the suffix \texttt{Request} or \texttt{Response}.
They are then used as input or output parameters of the corresponding gRPC function. \linebreak
This enables the implementation of the gRPC service.
Without proper implementation, the gRPC service will not be able to handle remote procedure calls or serialize and deserialize data properly.

% Describe the mapping with the application entity Customer from the API diagram AM-RentalManagementV1.0 to the corresponding methods of a gRPC service
A code snippet of \texttt{AM-RentalManagementV1.0}'s \texttt{.proto} file is shown in Listing \autoref{lst:proto_customer_service}.
The according API diagram is shown in \autoref{fig:ad_am-rental_management_v1.0}.
In the API diagram, the \texttt{Customer} application entity holds three functions and two attributes.
Therefore, the \texttt{Customer} application entity is mapped to a gRPC service named \texttt{CustomerService} holding three gRPC functions.
Here, each function is named after the corresponding method, with input and output parameters named as described above.
Each input and output parameter is defined in a single message holding the attributes of the original function.
Afterward, the \texttt{Customer} application entity is mapped to a message named \texttt{Customer} holding two attributes.
This concludes the derivation of the \texttt{Customer} application entity into the protobuf gRPC specification.

\begin{lstlisting}[
    style = kit-cm,
    caption = {Code Snippet of \texttt{AM-RentalManagementV1.0}'s \texttt{.proto} File},
    label = {lst:proto_customer_service}
]
//AM-RentalManagementV1.0
syntax = "proto3";

package rentalmanagement;

option go_package = "./pb";

import "google/protobuf/timestamp.proto";

...

service CustomerService{
    rpc RentCar(RentCarRequest) returns (RentCarResponse) {}
    rpc ListRentals(ListRentalsRequest) returns (ListRentalsResponse) {}
    rpc CancelRental(CancelRentalRequest) returns (CancelRentalResponse) {}
}

message RentCarRequest {
    google.protobuf.Timestamp startDate = 1;
    google.protobuf.Timestamp endDate = 2;
    Vin vin = 3;
    string customerId = 4;
}

message RentCarResponse {
    Rental rental = 1;
    ErrorDetail error = 2;
}

message ListRentalsRequest {
    string customerId = 1;
}

message ListRentalsResponse {
    repeated Rental rentals = 1;
    ErrorDetail error = 2;
}

message CancelRentalRequest {
    string rentalId = 1;
}

message CancelRentalResponse {
    bool result = 1;
    ErrorDetail error = 2;
}

message Customer {
    string customerId = 1;
    string name = 2;
}

...
\end{lstlisting}
\subsubsection*{Map Enumerations to gRPC Enums}

\subsubsection*{Translate Entities and Value Objects to gRPC Messages}

\subsubsection*{Transfer Parameters for gRPC Function}