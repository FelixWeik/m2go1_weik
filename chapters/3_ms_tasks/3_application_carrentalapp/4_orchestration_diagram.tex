\subsection{Excercise OrchestrationDiagramRentACar}
\subsubsection*{Orchestration Diagram Goals}
% What is the main goal of an orchestration diagram
The orchestration diagram is the dynamic counterpart of the API diagram.
It dynamically models use cases and thus the interactions between microservices.
The diagram is derived from the Use Cases with the help of the API specifications.

% From which artifact is the orchestration diagram "Rent a Car" derived?
The orchestration diagram "\texttt{Rent A Car}" is derived from the use case "\texttt{Rent a Car}", the API diagrams of the included microservices, and the API specifications of the included microservices, namely \texttt{AM-RentalManagement} and \texttt{DM-Car}.
The activity flow is derived from the use case "\texttt{Rent a Car}" as defined in the analysis phase.

\subsubsection*{API Calls}
% Where are the API calls located in the orchestration diagram
The API calls are located before calling each microservice.
The first call is started from the \texttt{starting point}, modeled as a full circle.
The starting point usually calls the application microservice.
% List all API calls, and explain the purpose of each
The calls are modeled according to the chosen API style and therefore the chosen microservice type:
\begin{description}
    \item[Application Microservices] are called using gRPC requests.
        In the orchestration diagram, the call is displayed by an arrow.
        The arrow is labeled with "\texttt{rpc}", the name of the function and the parameters without datatype and return value. 
    \item[Domain Microservices] are called using ReST requests.
        The corresponding call is modeled by adding the HTTP method and the endpoint.
    \item[External Systems] are added by the term "\texttt{Request}" and the required information derived from the information requirements.
\end{description}

% Differences between calls to AM-RentalManagement and DM-Car
The orchestration diagram \texttt{RentACarV1.0} is located in the GitLab instance \cite{CM-G-ORC}.
The explained differences can be seen here.
Calls to \texttt{AM-RentalManagement} are modeled as gRPC calls, having "\texttt{rpc}" as a label followed by the function's name and the parameters without datatypes.
Calls to \texttt{DM-Car} are modeled as ReST calls, being specified by the HTTP method and the exposed path.

\subsubsection*{Orchestration Between AM-Rentalmanagement and DM-Car}
% What triggers the orchestration diagram of "Rent a Car"

% Illustrate data flow between am-rentalManagement and dm-car in the orchestration

% What are possible responses from the orchestration diagram before termination

% How does UI-CarRental differentiate between the different responses
