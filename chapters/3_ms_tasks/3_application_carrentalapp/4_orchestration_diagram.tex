\subsection{Excercise OrchestrationDiagramRentACar}
\subsubsection*{Orchestration Diagram Goals}
% What is the main goal of an orchestration diagram
The orchestration diagram is the dynamic counterpart of the API diagram.
It dynamically models use cases and thus the interactions between microservices.
The diagram is derived from the Use Cases with the help of the API specifications.

% From which artifact is the orchestration diagram "Rent a Car" derived?
The orchestration diagram "\texttt{Rent A Car}" is derived from the use case "\texttt{Rent a Car}", the API diagrams of the included microservices, and the API specifications of the included microservices, namely \texttt{AM-RentalManagement} and \texttt{DM-Car}.
The activity flow is derived from the use case "\texttt{Rent a Car}" as defined in the analysis phase.

\subsubsection*{API Calls}
% Where are the API calls located in the orchestration diagram

% differences between calls to am-rentalManagement and dm-car

% List all API calls, and explain the purpose of each

\subsubsection*{Orchestration Between AM-Rentalmanagement and DM-Car}
% What triggers the orchestration diagram of "Rent a Car"

% Illustrate data flow between am-rentalManagement and dm-car in the orchestration

% What are possible responses from the orchestration diagram before termination

% How does UI-CarRental differentiate between the different responses
