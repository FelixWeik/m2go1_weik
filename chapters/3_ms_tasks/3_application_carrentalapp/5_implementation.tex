\subsection{Excercise API Controller}
\subsubsection*{Differences in the Micro Architecture}
% Explain Differences in Files between gRPC and ReST
Several differences appear in the file structure of the gRPC and ReST APIs.
First, differences in the subfolders appear.
The gRPC API has two subfolders, namely \texttt{pb} and \texttt{mappers}.
The ReST API has none of these, all files are located in the \texttt{controllers} folder.
Next, there are differences in the provided controller files.
While the ReST implementation has only one controller, the gRPC implementation has three namely the \texttt{CustomerController}, the \texttt{CustomersCollectionController}, and the \texttt{RentalsCollectionController}.
Next, the gRPC implementation has additional mappers.
Both implementations provide mappers in the \texttt{infrastructure} package mapping between persistency entities and models, yet the gRPC implementation provides additional mappers in the \texttt{controller} package mapping between serialized protobuf data and the models.
The gRPC implementation has the \texttt{pb} subfolder containing generated code from the specification.
It holds the primitives to interact with the API that are located in the generated code of the ReST implementation.
Also, the generated files of the ReST implementation are much shorter than in the gRPC implementation.

\subsubsection*{Interfaces Implementation}
% Which interfaces are implemented by CustomerController and RentalsCollectionController
The \texttt{CustomerController} implements the \texttt{CustomerOperationsInterface} provided by the \texttt{operations} package.
The struct also contains the \texttt{UnimplmentedCustomerServiceServer} provided by the \texttt{pb} package in the \texttt{controller/pb} subfolder.
The \texttt{RentalsCollectionController} implements the \texttt{RentalsCollectionOperationsInterface} also provided by the \texttt{operations} package.
The according struct contains the \texttt{UnimplementedRentalsCollectionServiceServer} provided by the \texttt{pb} package in the \texttt{controller/pb} subfolder as well.

% How do the interfaces relate to the protobuf specification
Both interfaces relate to the protobuf specification.
The \texttt{CustomerOperationsInterface} contains the functions specified in the \texttt{CustomerService} of the protobuf specification.
The specified input and output parameters are defined for the functions as well.
The \texttt{RentalsCollectionOperationsInterface} contains the functions specified in the \texttt{RentalsCollectionService} of the protobuf specification.
Here too, the specified input and output parameters are defined for the functions.

% Why does the CustomerController contain the entry pb.UnimplmentedCustomerServiceServer
The \texttt{api_specification_am_rental_management_grpc.pb.go} contains two types of CustomerServiceServer: \texttt{CustomerServiceServer} and \texttt{UnimplementedCustomerServiceServer}.
While the first one is an interface, the second one is a struct, yet both contain the functions implementing the interface.
The \texttt{UnimplementedCustomerServiceServer} contains the functions that throw an error, telling the user that the function is not implemented yet.
Therefore, the \texttt{UnimplementedCustomerServiceServer} stubs the interface until the interface is finally implemented.
Every time the function is called, the error is thrown, and the program will exit safely.
Thus, the \texttt{CustomerController} contains the \texttt{UnimplementedCustomerServiceServer} to stub the interface.

\subsubsection*{Mappers}
% Why are there mappers between the API controller and the logic

% Which data is transformed
