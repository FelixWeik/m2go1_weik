\subsection{Excercise LocalDeployment}
\subsubsection*{Containerize AM-RentalManagement}
% Why can the dockerfile be reused?
Dockerfiles are scripts automating the process of building Docker images.
A Dockerfile defines the base image, building and testing steps and the entry point of a Docker image.
% Why is this Docker Image reusable?
In the case of AM-RentalManagement and DM-Car building and testing commands are the same since both projects are implemented using a similar file structure and the same programming language.
Therefore commands like \texttt{go build -o main .} or \texttt{go test -v ./...} can be reused.

% Conditions and characteristics that make Dockerfiles reusable across projects
This shows, that several characteristics have to be met to reuse a Dockerfile across projects.
The applications should:
\begin{itemize}
    \item be implemented using the same programming language.
    \item be implemented using a similar file structure.
    \item share the same dependencies.
    \item be built and tested using the same commands.
    \item be deployed using the same commands.
    \item be deployed using the same entry point.
\end{itemize}

% How does this influence development workflows
Several advantages arise from reusing Dockerfiles across projects.
First, it guarantees consistency across projects, since the same base image and building steps are used.
This enables time and resource savings since the Dockerfile is already implemented and tested for a certain use case.
Next, it reduces the complexity of the Dockerfile, since it is not necessary to implement the same steps again.
This reduces the risk of errors and increases the readability of the Dockerfile.
Finally, it reduces the maintenance effort, since the Dockerfile only needs to be maintained in one place.
Therefore, reusing a Dockerfile if it is possible is a good practice.

% \subsubsection*{Orchestrate Using Docker Compose}
% % How does Docker Compose automatically allow services to communicate with each other using service names as host names
