\chapter{Special Contributions}
\label{ch:special_contributions}

\section{Go Environment Variables}
This section is executed on an unix-based system, in the author's case it's Ubuntu.

\subsection*{Installation}
During the go installation, one needs to add paths to the environment variables.
To add the paths add \texttt{export PATH\=\$PATH\:\/usr\/local\/go\/bin} 
to the \texttt{\$HOME/.profile} for user-wide installation or to \texttt{/etc/profile} for a system-wide installation.
But what is the difference between both files and what exactly does this extra variable in the file do?

\subsubsection*{The .profile File}
This file is used to set environment variables and configurations for the shell.
It contains the file paths for commands the system will check for instead of the user typing the full path to the file everytime he wants to execute it.
When a user logs into the system, the file is executed to initialize the environment for the user's session.
The file is located in the user's home directory and is hidden by default.

The usual setup is separated into two files:
\begin{enumerate}
    \item System-wide configuration:
        Files like \texttt{/etc/profile} are used to set up the environment for all users.
        These files are executed when any user logs in.
    \item User-specific configuration:
        Each user has their own profile configuration file.
        This file is located in the user's home directory like \texttt{\$HOME/.profile} or \texttt{\$HOME/.bashrc}.
        These files are executed when the specific user logs in.
\end{enumerate}

\subsubsection*{The PATH Variable}
If one does not include a \texttt{PATH} variable in her system, the shell will only search for executable files in the system-wide paths defined in the default \texttt{PATH} variable.
If the binary is not located in these system-wide paths, the file will not be able to be executed correctly without providing the complete path.
Adding a directory to the \texttt{PATH} allows the shell to search in custom directories for executable (or binary) files.
Therefore, commands and programs can be executed without specifying the full path.

In the example above, the \texttt{PATH} variable is extended by the path to the go binaries, enabling go files in the \texttt{PATH} to be executed without specifying the full path.
